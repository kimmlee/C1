%!TEX root = JournalChapter1.tex
\section{Introduction}
\label{introduction}

Microblogs have grown in popularity in recent years, gradually transforming the way we find out about the latest events and communicate. Twitter is the most prominent service \footnote{\url{https://twitter.com/}}, as it is used by millions, posting over 340 million tweets every day\footnote{\url{http://blog.twitter.com/2012/03/twitter-turns-six.html}}. Microblog services are used for various purposes including: (i) self promotion, (ii) advertising, (iii) real-time news broadcasting, (iv) social discussions etc. The most important aspect of Twitter is that it provides unique insight into real-time events, such as first hand reports of events as they are developing, along with the opinion of those discussing them.
This information makes Twitter a uniquely valuable media source, which led to obtaining much attention by research and industrial communities.

Ad-hoc retrieval is one of the most widely investigated tasks in Information Retrieval (IR) where the goal is to return documents that are relevant to an immediate information need. Recently ad-hoc retrieval has been actively studied in the context of microblogs, particularly during the microblog tracks at TREC 2011, 2012 and 2013~\cite{ounis2011overview}. 

Searching Twitter can be extremely challenging because of document morphology. Their content is limited as messages posted to Twitter (known as \emph{Tweets}) are limited to 140 characters in length. Furthermore they are generally of a varied linguistic quality \cite{teevan2011twittersearch} due to colloquialisms and users attempting to fit the content within the message limitations. 

More specifically tweets pose new constraints for which state of the art retrieval models were not designed for\footnote{Models such as: Okapi BM25~\cite{robertson2009probabilistic}; Divergence From Randomness (DFR)~\cite{amati2003probabilistic}; Hiemstra's Language Model (HLM)~\cite{model}; and Dirichlet Language Model (DLM)~\cite{zhai2001study}}. To the best of our knowledge, nobody has properly assessed the effect that these new characteristics or dimensions have on the current relevance assumptions in which retrieval models are based.

Whilst few recent works have identified some features as possibly being detrimental in microblog ad-hoc retrieval \cite{singhal1996pivoted,naveed2011searching}, no study has been carried out to determine the concrete effect of the different features on state of the art retrieval models. The main objective of this work is not the improvement over the very best scores achieved for each of the microblog collections taking into consideration all available approaches to the day. Instead we are set to investigate the connection of the structure of microblog documents with their relevance during an ad-hoc search task. This work, addresses the following question: What are the reasons behind the bad performance of state of the art retrieval models, in the context of microblogs?.

To this end , firstly we observe the performance of state of the art retrieval models in the context of Twitter corpora selecting the best retrieval model as a baseline. Then we perform a series of experiments which simulate the behaviour of a number of state of the art retrieval models in order to identify possible shortcomings in their design with respect to microblog documents. This initial experiment is completed with the creation of a retrieval model which takes into account all previous findings, namely MBRM. MBRM demonstrates that the scope hypotheses still holds within microblog documents, and that microblog document statistics can be leveraged to significantly improve ad-hoc retrieval performance.

Secondly, we study the behaviour of inherent features and characteristics to microblog documents and evaluate their suitability for enhancing the behaviour of state of the art retrieval models. Moreover, a number of experiments are performed to demonstrate which microblog specific features are most indicative of the relevance of microblogs by demonstrating statistically significantly improved retrieval performance for ad-hoc search.

Finally, we extend our analysis by considering the ordering of the different component that make up microblog documents. In order to do so, we encode structure of observed relevant and non-relevant documents in state machines, which in turn are used to produce scores for re-ranking. We utilise the 2013 microblog collection to construct such the state machines, and test our approach on the 2011 and 2012 microblog collections combined. Our results show statistically significantly improved results over the selected baseline, demonstrating the connection of microblog structure with their relevance.

With the objective of studying microblog document's structures, we set the focus of this work in the context of these research questions: 

\begin{itemize}

\item[] \textbf{RQ1.} Are there structural differences between relevant and non-relevant microblog documents? Can we exploit them for ad-hoc retrieval?

\item[] \textbf{RQ1.1} Does document length have any connection with the relevance of microblog documents?
\item[] \textbf{RQ1.2} Does term frequency of query terms relate to the relevance of microblog documents? 
\item[] \textbf{RQ1.3} Can we adapt state of the art retrieval models to better handle microblog documents?
\item[] \textbf{RQ1.4} Can we devise a retrieval model to better capture the relevance of microblog documents?

\item[] \textbf{RQ2.} Can microblog features be exploited to help retrieval models better capture relevance than current retrieval models?
\item[] \textbf{RQ3.} Is the order of the different elements in a microblog document connected with relevance? Can it be utilised for ad-hoc retrieval?

\end{itemize}

The rest of the chapter is organised as follows. First, we cover some of the relevant literature regarding microblog searches and introduce the concepts utilised throughout this work (Section \ref{background}). Section \ref{experiment} sets the evaluation environment in which our investigation is carried out, giving way to our main analysis (Section \ref{discussion}). Finally Section \ref{conclusion} concludes the work and indicates future research directions.

%\mentalnote{ What's microblog }
%
%how they differ from web documents and blogs
%
%problems in retrieval
%
%This document will be driven by the following research questions:
%
%\begin{itemize}
%\item Does the retrieval model assumption hold for microblog documents?
%\item Can we model the informativeness of a microblog document?
%\item Is the structure of Microblog documents related to its relevance?
%\end{itemize}


