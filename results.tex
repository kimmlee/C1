%!TEX root = ../thesis.tex
%\section{Results}
%
%In this section we present the results obtained for each of the approaches we are evaluating and the baseline.
%
%\begin{table}
%\caption{Evaluation metrics for state of the art retrieval models evaluated over the 2011 and 2012 microblog collections}
%\begin{small}
%\hspace{1cm}
%
%  \begin{tabular}{l||c|c|c|c|c|c|} 
%
%& BM25 &	DFR &	HLM &	IDF &	TF\_IDF  & Dirich. \\
%\hline
%P\_5 & 0.477  &	0.483 &	0.377&	\textbf{0.505}	&0.477	&0.463\\
%P\_10  &   0.430&	0.454	&0.351	&\textbf{0.462}	&0.437&	0.407\\
%P\_15  &  0.416 &	0.428	&0.335	&\textbf{0.435}&	0.416	&0.382\\
%P\_20  &   	0.385&	0.401	&0.312&	\textbf{0.413}	&0.385	&0.355\\
%P\_30 & 0.344&	0.363	&0.279	& \textbf{0.377}&	0.344&	0.317\\
%\hline
%\end{tabular}
%
%\label{sota}
%\end{small}
%\end{table}




%\begin{figure}[]
%  \caption{Exploration of Logarithmic discounting function base for Query Expansion. (Blue area shows best performance across precision cutoffs)}
%  \centering
%    \includegraphics[width=0.5\textwidth]{logbase}
%	\label{logarithm}
%\end{figure}


%\begin{tikzpicture} \begin{axis}[ nodes near coords={(\coordindex)}, footnotesize, title={patch refines=\level}] \addplot3[patch,patch type=triangle quadr, shader=faceted interp,patch refines=\level] coordinates { (0,0,0) (5,4,0) (0,7,0) (2,3,0) (3,6,1) (-1,4,0) }; \end{axis} \end{tikzpicture}
%


