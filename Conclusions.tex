%!TEX root = JournalChapter1.tex
\section{Conclusions}

\label{conclusion}

In this work, we verified whether the scope and verbosity hypotheses still hold for microblog document retrieval. We initially hypothesise that since microblog documents have a character limit the scope and verbosity hypotheses could not hold, as it is assumed that the author of the document is able to produce documents of any length. 

We then proceeded to analyse the behaviour of a number of state of the art retrieval models. The models chosen were BM25, HLM, DLM, DFRee and IDF. Our experimentation led to a better understanding of what could be the shortcomings experienced by such models under microblog retrieval constraints. Particularly, we isolated the fact that longer documents should be promoted to account for effort of microblog authors to encode their messages into the character limit. Then we identified that higher term frequencies than 1-2 should be penalised as they are more likely to be less informative and more reminiscent of spam. Based on these observations we concluded that the scope hypotheses does hold in microblog retrieval, however verbosity does not.

Finally we built a retrieval model optimised for microblog retrieval, namely MBRM, which significantly outperforms the best baselines, by making better use of document-encoded evidence.

Future work will demonstrate how MBRM can be used to push further the current performance of approaches that rely on the initial results such as Automatic Query Expansion.\\

